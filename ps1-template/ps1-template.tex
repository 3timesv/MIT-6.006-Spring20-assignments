%
% 6.006 problem set 1 solutions template
%
\documentclass[12pt,twoside]{article}

\input{macros-sp20}
\newcommand{\theproblemsetnum}{1}

\title{6.006 Problem Set 1}

\begin{document}

\handout{Problem Set \theproblemsetnum}

\setlength{\parindent}{0pt}
\medskip\hrulefill\medskip

{\bf Name:} Vivek Verma

\medskip

{\bf Collaborators:} None

\medskip\hrulefill

%%%%%%%%%%%%%%%%%%%%%%%%%%%%%%%%%%%%%%%%%%%%%%%%%%%%%
% See below for common and useful latex constructs. %
%%%%%%%%%%%%%%%%%%%%%%%%%%%%%%%%%%%%%%%%%%%%%%%%%%%%%

% Some useful commands:
%$f(x) = \Theta(x)$
%$T(x, y) \leq \log(x) + 2^y + \binom{2n}{n}$
% {\tt code\_function}


% You can create unnumbered lists as follows:
%\begin{itemize}
%    \item First item in a list
%        \begin{itemize}
%            \item First item in a list
%                \begin{itemize}
%                    \item First item in a list
%                    \item Second item in a list
%                \end{itemize}
%            \item Second item in a list
%        \end{itemize}
%    \item Second item in a list
%\end{itemize}

% You can create numbered lists as follows:
%\begin{enumerate}
%    \item First item in a list
%    \item Second item in a list
%    \item Third item in a list
%\end{enumerate}

% You can write aligned equations as follows:
%\begin{align}
%    \begin{split}
%        (x+y)^3 &= (x+y)^2(x+y) \\
%                &= (x^2+2xy+y^2)(x+y) \\
%                &= (x^3+2x^2y+xy^2) + (x^2y+2xy^2+y^3) \\
%                &= x^3+3x^2y+3xy^2+y^3
%    \end{split}
%\end{align}

% You can create grids/matrices as follows:
%\begin{align}
%    A =
%    \begin{bmatrix}
%        A_{11} & A_{21} \\
%        A_{21} & A_{22}
%    \end{bmatrix}
%\end{align}

% You can include images and PDFs as follows:
% \includegraphics[width=0.5\textwidth]{img.jpg}

\begin{problems}

\problem  % Problem 1

\begin{problemparts}
\problempart % Problem 1a
$(f_5, f_3, f_4, f_1, f2)$
\problempart % Problem 1b
$(f_1, f_2, f_5, f_4, f_3)$
\problempart % Problem 1c
$(\{f_2, f_5\}, f_4, f_1, f_3)$
\problempart % Problem 1d
$(f_5, f_3, f_4, f_2, f_1)$ \quad WRONG \\
$(f_5, f_2, f_1, f_3, f_4)$ \quad From solution
\end{problemparts}

\newpage
\problem  % Problem 2

\begin{problemparts}
\problempart % Problem 2a
\problempart % Problem 2b
\end{problemparts}

\newpage
\problem  % Problem 3

\newpage
\problem  % Problem 4

\begin{problemparts}
\problempart % Problem 4a
$insert\_first(x)$ \\
Create a Node containing the item x. \\
Check the head of the list to see if list is empty. \\
If list is empty, set the Node as head and tail of the list and return. \\
If list is non-empty, get the head of the list called $first$, Connect $first$ as next element of $x$ and $x$ as previous element of $first$. \\ 
Set the head of the list to $x$. \\ \\
$insert\_last(x)$ \\
Create a Node containing the item x. \\
Check the tail of the list to see if list empty. \\
If list is empty, set the Node as head and tail of the list and return. \\
If list is non-empty, get the tail of list called $last$,
Connect $last$ as previous element of $x$ and $x$ as next element of $last$.\\
Set the tail of the list to $x$. \\ \\

$delete\_first()$ \\
Get the first two elements as $first$ and $second$. \\
Set $second$ as the head of the list and set $second.prev$ as None. \\
Return $first.item$. \\ \\ 

$delete\_last()$ \\
Get the last two elements as $last$ and $second\_last$. \\
Set $second\_last$ as tail of the list and set $second\_last.next$ as None. \\
Return $last.item$. \\

\newpage
\problempart % Problem 4b
Construct a new empty list $L2$. \\
There are four cases based on the location of $x_1$ and $x_2$ in the list: \\
\begin{enumerate}
    \item Neither $x_1$ is head nor $x_2$ is tail: \\
    \hspace*{3em} Connect the element before $x_1$ to the element after $x_2$.
    \item $x_1$ is the head: \\
    \hspace*{3em} Set the head of the list to the element after $x_2$ and set the previous element of head as None.
    \item $x_2$ is the tail: \\
    \hspace*{3em} Set the tail of the list to the element before and set the next element of tail as None.
    \item $x_1$ is the head and $x_2$ is the tail: \\
    \hspace*{3em} Set the head and tail of the list to None.
\end{enumerate}

Independent of which case above was executed, \\
Set the previous element of $x_1$ and next element of $x_2$ as None. \\
Set the head of $L2$ to $x_1$ and the tail of $L2$ to $x_2$. \\

\problempart % Problem 4c
If $L2$ is an empty list, nothing needs to be done, hence return None. \\
Otherwise, \\
Get the element after $x$ in a variable $x\_next$. \\
Set the next element of $x$ to head of $L2$ and previous element of $L2.head$ to $x$. \\

If $x\_next$ is not None, Set previous element of $x\_next$ to tail of $L2$ and next next element of $L2's$ tail to $x\_next$. \\
Set the tail of $L$ as tail of $L2$. \\
Set head and tail of $L2$ to None. \\
\newpage
\problempart 
\\
\begin{lstlisting}
    def insert_first(self, x):
        # create node with item x
        x = Doubly_Linked_List_Node(x)

        # if list is empty, add x as only element
        if self.head is None:
            self.head = x
            self.tail = x
            return

        # get the first element and connect it to x
        first = self.head
        first.prev = x

        # connect x to first
        x.next = first

        # set x as the head
        self.head = x

    def insert_last(self, x):
        # create node with item x
        x = Doubly_Linked_List_Node(x)

        # if list is empty add, x as the only element
        if self.tail is None:
            self.tail = x
            self.head = x

        # get the last element and connect it to x
        last = self.tail
        last.next = x

        # connect x to last element
        x.prev = last

        # set x as the tail
        self.tail = x

    def delete_first(self):
        # get first two elements
        first = self.head
        second = first.next

        # set second element as head ot the list
        second.prev = None
        self.head = second
        return first.item

    def delete_last(self):
        # get last two elements
        last = self.tail
        second_last = last.prev

        # set second last element as tail of the list
        second_last.next = None
        self.tail = second_last
        return last.item

    def remove(self, x1, x2):
        L2 = Doubly_Linked_List_Seq()

        # Neither x1 nor x2 is the head or tail
        if x1.prev is not None and x2.next is not None:
            # connect the element before x1 to the element after x2
            x1.prev.next = x2.next
            x2.next.prev = x1.prev
        # if x1 is the head but x2 is not tail
        elif x1.prev is None and x2.next is not None:
            # disconnect elements x1 to x2 from the list
            self.head = x2.next
            self.head.prev = None
        # if x2 is the tail but x1 is not head
        elif x2.next is None and x1.prev is not None:
            # disconnect elements from x1 to x2 from the list
            self.tail = x1.prev
            self.tail.next = None
        # x1 is head and x2 is tail
        else:
            self.head = None
            self.tail = None

        # disconnect the links to any element before x1 and any elment
        x1.prev = None
        x2.next = None

        # set the head and tail of L2
        L2.head = x1
        L2.tail = x2
        return L2

    def splice(self, x, L2):
        # if L2 is empty, return None
        if L2.head is None:
            return
        
        # get the element after x
        x_next = x.next

        # connect x to head of L2
        x.next = L2.head
        L2.head.prev = x

        # there was an element after x, connect it to tail of L2
        if x_next is not None:
            x_next.prev = L2.tail
            L2.tail.next = x_next

        # set the tail of current list as tail of L2
        self.tail = L2.tail

        # remove all elements from L2
        L2.head = None
        L2.tail = None
\end{lstlisting}
\end{problemparts}

\end{problems}

\end{document}
