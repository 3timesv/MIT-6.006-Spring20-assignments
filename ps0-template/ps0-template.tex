%
% 6.006 problem set 0 solutions template
%
\documentclass[12pt,twoside]{article}

\input{macros-sp20}
\newcommand{\theproblemsetnum}{0}

\title{6.006 Problem Set 0}

\begin{document}

\handout{Problem Set \theproblemsetnum}

\setlength{\parindent}{0pt}
\medskip\hrulefill\medskip

{\bf Name:} Vivek Verma

\medskip\hrulefill

%%%%%%%%%%%%%%%%%%%%%%%%%%%%%%%%%%%%%%%%%%%%%%%%%%%%%
% See below for common and useful latex constructs. %
%%%%%%%%%%%%%%%%%%%%%%%%%%%%%%%%%%%%%%%%%%%%%%%%%%%%%

% Some useful commands:
% $f(x) = \Theta(x)$
% $T(x, y) \leq \log(x) + 2^y + \binom{2n}{n}$
% \ttt{code\_function}


% You can create unnumbered lists as follows:
% \begin{itemize}
%     \item First item in a list
%         \begin{itemize}
%             \item First item in a list
%                 \begin{itemize}
%                     \item First item in a list
%                     \item Second item in a list
%                 \end{itemize}
%             \item Second item in a list
%         \end{itemize}
%     \item Second item in a list
% \end{itemize}

% You can create numbered lists as follows:
% \begin{enumerate}
%     \item First item in a list
%     \item Second item in a list
%     \item Third item in a list
% \end{enumerate}

% You can write aligned equations as follows:
% \begin{align}
%     \begin{split}
%         (x+y)^3 &= (x+y)^2(x+y) \\
%                 &= (x^2+2xy+y^2)(x+y) \\
%                 &= (x^3+2x^2y+xy^2) + (x^2y+2xy^2+y^3) \\
%                 &= x^3+3x^2y+3xy^2+y^3
%     \end{split}
% \end{align}

% You can create grids/matrices as follows:
% \begin{align}
%     A =
%     \begin{bmatrix}
%         A_{11} & A_{21} \\
%         A_{21} & A_{22}
%     \end{bmatrix}
% \end{align}

\begin{problems}

\problem  % Problem 1

\begin{problemparts}
\problempart % Problem 1a
\{6, 12\}
\problempart % Problem 1b
7
\problempart % Problem 1c
2
\end{problemparts}

\problem  % Problem 2

\begin{problemparts}
\problempart % Problem 2a
1.5
\problempart % Problem 2b
12.25
\problempart % Problem 2c
13.75
\end{problemparts}

\problem  % Problem 3

\begin{problemparts}
\problempart % Problem 3a
True
\problempart % Problem 3b
False
\problempart % Problem 3c
False
\end{problemparts}

\newpage
\problem  % Problem 4

\textbf{Induction Hypothesis: }

\[\sum_{i=1}^{n} i^3= \left[ \frac{n(n+1)}{2}} \right]^2  \quad \forall \quad n >= 1\]


\textbf{Base case:} \quad $n=1$ \\
\begin{align*}
    L.H.S.:& \quad 1^3 = 1  \\
    R.H.S.:& \quad \left(\frac{1(1+1)}{2}\right)^2 = 1
\end{align*}
\\
$\therefore$ Base case is true. \\ \\
{\bf Inductive step: } \\
Assume the hypothesis is true for $n$ \\
i.e 
\[\sum_{i=1}^{n} i^3= \left[ \frac{n(n+1)}{2}} \right]^2\]

Adding $(n+1)^3$ on both sides,
\begin{align*}
    \sum_{i=1}^{n} i^3 + (n+1)^3 &= \left[\frac{n(n+1)}{2}\right]^2 + (n+1)^3 \\
    &= \frac{n^2(n+1)^2}{4} + (n+1)(n+1)^2 \\
    &= (n+1)^2 \left[\frac{n^2 + 4(n+1)}{4}\right] \\
    &= \frac{(n+1)^2 (n+1)^2}{4} \\
    &= \left[\frac{(n+1)(n+2)}{2}\right]^2 \\
    \implies \sum_{i=1}^{n+1} i^3 &= \left[\frac{(n+1)((n+1)+1)}{2}\right]^2
\end{align*}
\\
which proves that hypothesis is true for $n+1$. \\
So, it follows by induction that the hypothesis is true $\forall n >= 1$. \qed
\newpage
\problem  % Problem 5
\\ 
{\bf Induction Hypothesis: } Every connected undirected graph $G = (V, E)$ for which $|V| = n$ and $|E| = n-1$ is acyclic. \\

{\bf Base case: } Consider, 
\[n = 3,\] 
\[|E| = 3-1 = 2\]

If the three vertices are A, B, and C, Then the two edges will be a set of any two edges from the set $\{A-B, B-C, C-A\}$. \\
$\therefore$ Base case is true.
\\

{\bf Inductive step: } Assume a graph $G$ with $n$ vertices and $(n-1)$ edges is acyclic. \\

For any pair of vertices $(v_1, v_2)$, there is a path between $v_1$ and $v_2$ which may or may not contain other vertices in between. \\

To add a new vertex $v_0$ in the graph such that it results in graph $G'$ containing $(n+1)$ vertices and $n$ edges, there are 3 options:
\begin{enumerate}
    \item Connect $v_0$ to $v_1$
    \item Connect $v_0$ to $v_2$
    \item Connect $v_0$ to any vertex which lies on the path between $v_1$ and $v_2$
\end{enumerate}

After adding the new vertex through any of the option out of above 3, the graph remains acyclic because one end of vertex is not connected to any other vertex. \\ \\
Therefore, the hypothesis is true for graph $G`$. \\
Hence, it follows by induction that every connected undirected graph $G = (V, E)$ for which $|E| = |V| - 1$ is acyclic. \qed 

\newpage
\vfill
\problem  % Problem 6
Submit your implementation to {\small\url{alg.mit.edu}}.

\begin{lstlisting}
def count_long_subarray(A):
    '''
    Input:  A     | Python Tuple of positive integers
    Output: count | number of longest increasing subarrays of A
    '''
    count = 0
    
    # maximum length of increasing sub-array
    max_length = 1
    # length of current increasing sub-array
    sub_length = 1

    for i in range(len(A)):
        # if next element is greater than current element, increment sub_length
        if i < len(A)-1 and A[i] < A[i+1]:
            sub_length += 1
        else:
            # if current sub-array has more length than previous maximum sub-array
            if max_length < sub_length:
                # re-initialize maximum length and count
                max_length = sub_length
                count = 1
            # if current sub-array has same length as previous maximum sub-array,
            elif max_length == sub_length:
                # increment count
                count += 1
            # reset sub-array length to 1
            sub_length = 1

    # compare length of last sub-array with maximum length sub-array
    if max_length < sub_length:
        max_length = sub_length
        count = 1
    elif max_length == sub_length:
        count += 1
        
    return count
\end{lstlisting}

\end{problems}

\end{document}
